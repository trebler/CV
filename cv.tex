\documentclass[11pt,a4paper]{moderncv}
\usepackage[utf8]{luainputenc}
\setcounter{secnumdepth}{2}
\setcounter{tocdepth}{2}

\makeatletter

\moderncvstyle{classic}
\moderncvcolor{black}
\moderncvicons{awesome}

\usepackage[scale=0.84]{geometry}
\setlength{\hintscolumnwidth}{2.9cm}
\AtBeginDocument{\recomputelengths}

% personal data
\title{Curriculum Vit\ae{}}
\name{Andrey}{Trebler}
\phone[mobile]{+47 45100719}
\email{andrey@trebler.dev}
\address{Vestre Helleveien 5B}{4318 Sandnes}{Norway}
\social[linkedin]{andreytrebler} 
\social[github]{trebler} 
\photo[75pt][0.4pt]{photo.png}

%\nopagenumbers{}

\usepackage{lastpage}
\rfoot{\addressfont\itshape\textcolor{gray}{Page \thepage\ of \pageref{LastPage}}}
\lfoot{\addressfont\itshape\textcolor{gray}{ver.: \directlua{tex.print(os.getenv("VERSION") or "unknown")}}}

\makeatother

\begin{document}
\makecvtitle

\section{Personal Information}
%\cvitem{Family name}{\emph{Trebler}}
%\cvitem{First name}{\emph{Andrey}}
\cvitem{Date of birth}{\emph{05/04/1986}}
\cvitem{Citizenship}{\emph{Norwegian}, originally from \emph{Kazakhstan}}
\cvitem{Marital status}{\emph{Married}, \emph{father of 3}}

%\section{Professional Interests} 
%\cvitem{}{Software development; Mathematical modeling and simulations; control systems and automation; computer graphics}

\section{Key Qualifications} 
% \cvitem{}{Software development; mathematical modeling and numerical calculations; simulation software and real-time simulators; computer graphics}
\cvitem{}{Experienced *nix-oriented developer with passion for backend, system design, and design of APIs; interest in development of rich reactive web applications; and fun of using advanced features of Kubernetes for modern containerized deployment}

\section{Work Experience}
\cventry{01/2019--\ldots}{Senior Software Engineer}{eDrilling}{Stavanger, Norway}{}{Team lead and software architect responsible for transition to cloud and design of modern APIs. Development and deployment of higly available and scalable Kubernetes-native systems based on microservices architectecture. Design and implementation of HTTP REST(ful), SSE and WebSocket APIs. Development of reactive web applications\newline{}%
\begin{itemize}%
\item Backend: \texttt{TypeScript/node.js (ES2021)}, \texttt{Go}, \texttt{Python}, \texttt{C++17/Qt}
%\item Desktop: \texttt{C++17/Qt}
\item Frontend: \texttt{TypeScript/Vue.js (2/3)}, \texttt{TypeScript/React}
\item DevOps: \texttt{Docker}, \texttt{Kubernetes}, \texttt{Azure}, \texttt{AWS}
\item Technologies: \texttt{Redis}, \texttt{RabbitMQ/AMQP}, \texttt{WebSockets}, \texttt{OpenAPI}, \texttt{Traefik}, \texttt{Keycloak}, \texttt{Azure AD}
\end{itemize}}
\cventry{10/2016--01/2019}{Full Stack Developer}{timeanddate.com}{Stavanger, Norway}{}{Development of numerical algorithms, APIs, and software for astronomical simulations\newline{}%
\begin{itemize}% 
\item Backend development (in \texttt{C99}) of internal and external APIs providing astronomical data
\item Frontend development (in \texttt{JavaScript}  using \texttt{D3} and \texttt{three.js} frameworks) of browser-based applications for astronomical visualization. \texttt{ECMAScript 5}, \texttt{CSS/SASS} and \texttt{HTML5}
\end{itemize}}
\cventry{03/2016--10/2016}{Software Engineer}{Steinsvik}{F{\o}rresfjorden}{R\&D Department}{Development of software for fish farming industry \newline{}%
\begin{itemize}% 
\item Development (frontend and backend) of cross-platform software for camera control and monitoring at fish farms in \texttt{Qt/QML} and \texttt{C++11}
\item Support of existing software implemented in \texttt{NI LabVIEW}
%\item Research and testing of new technologies for data transfer; pattern recognition; optical lenses
\end{itemize}}
\cventry{12/2013--02/2016}{Simulator Systems Analyst}{MHWirth}{Stavanger}{Simulators Department}{Development of real-time simulators for oil and gas industry \newline{}%{Development of real-time simulators (test/pre-commissioning, training, HIL [Hardware-in-the-Loop]) for oil and gas industry \newline{}%
\begin{itemize}% 
\item Development of mathematical models for equipment simulation %in NI LabVIEW, Simatic Step~7, \texttt{C\#}, SimulationX
\item Control system software implementation
\item HIL testing of control system software
%\item Testing of in-house simulation software, bug reporting 
%\item Implementation of projects using SCRUM methodology. User stories preparation%. Acceptance criteria definition and verification
%\item Implementation of interfaces with control systems and HMI/SCADA systems
%\item Integration of control systems, equipment models, instrument models, I/O interfaces and 3D visualization
%\item Application of in-house downhole model to research and training projects
\end{itemize}}
% \cventry{06/2010--06/2013}{PhD Research Fellow (\emph{doktorgradsstipendiat})}{University of Oslo---UiO}{Oslo}{Department of Geosciences, Section for Meteorology and Oceanography}{Research on project \emph{Sources of Greenhouse Gases in East Asia (SOGG--EA)}\newline{}%
%funded by Research Council of Norway \newline{}%
%\begin{itemize}%  
%\item Development of mathematical models (in \texttt{Fortran} and \texttt{C/C++}) for determining emission sources
%\item Data assimilation, analysis and visualisation (MATLAB, OpenGL)
%\end{itemize}
% }
\cventry{10/2007--01/2008}{Software Test Engineer}{ABBYY}{Moscow}{Mobile software testing group}{Gray-box testing of applications for Symbian and Windows Mobile mobile operating systems \newline{}%
%\begin{itemize}%
%\item Manual and semi-automatic testing of software on mobile devices
%\item Ticket control system fulfilment
%\item Agile software development practices
%\end{itemize}
}
%\cventry{09/2007--05/2009}{Linguist}{ABBYY}{Moscow}{``Parallel resources'' group}{Alignment (matching) of diverse texts in English and Russian for creating multilingual corpus for universal electronic translator using ABBYY Aligner software tools}

%\section{Project involvement}
%\cvitemwithcomment{Snorre A}{Statoil (Stavanger)}{HIL simulator used for testing control systems and running training courses. Simulation, I/O communication, 3D configuration, implementation of HIL tester UI. Execution of HIL testing and testing of sequences for training}
%\cvitemwithcomment{WOCS (Workover Control System)}{Aker Subsea (Ågotnes)}{Training/HIL simulator for WOCS. Complete hydraulic simulation of hydraulic power unit and subsea modules, I/O communication, integration with WOCS SCADA system, implementation of training scenarious (hydraulic faults)}
%\cvitemwithcomment{Opus Tiger}{Opus Offshore (Shanghai)}{Training simulator for Opus Tiger Drillship. Implementation of control systems based on specifications and functional descriptions from Hong Hua. Simulation and configuration of 3D scene. Integration with MHWirth control and HMI systems.}
%\cvitemwithcomment{EKD (Enhanced Kick Detection)}{Statoil (Rotvoll)}{Research-oriented simulator based on MHWirth's Cat-D simulator for purpose of developing kick detection algorithm. Implementation and integration with detailed mudpump simulator, downhole simulator and 3rd parth kick detection algorithm}
%\cvitemwithcomment{Aker Spitsbergen}{Transocean (Kristiansand)}{Upgrade of drilling training simulator}
%\cvitemwithcomment{Songa Equinox Cat-D}{Songa (Kristiansand)}{Upgrade of drilling training simulator to latest version of control systems software, HMI software and 3D visualization. Implementation of simulations and addition of training-oriented fault scenarios}
%\cvitemwithcomment{DHS (Downhole Training Simulator)}{BP (Baku)}{Upgrade of drilling training simulator. Integration with in-house downhole simulator}
%\cvitemwithcomment{Visioneer}{MHWirth (Stavanger)}{Integration of drilling simulator with new in-house 3D visualization and simulation tool. Testing of Visioneer and bug reporting. Preparation of user stories and acceptance criteria for SCRUM process}
%\cvitemwithcomment{iMPD (Managed Pressure Drilling)}{Statoil (Rotvoll)}{Integration of simulator with downhole model developed by IRIS (International Research Institute Of Stavanger)}

\newpage{}

\section{Education}
\cventry{06/2010--06/2013}{PhD Program in Atmospheric sciences}{University of Oslo---UiO}{Faculty of Mathematics and Natural Sciences}{Department of Geosciences}{Section for Meteorology and Oceanography}
\cventry{10/2008--05/2010}{PhD Program in Mathematical Modelling, Numerical Methods and Programming}{Lomonosov Moscow State University}{Faculty of Computational Mathematics and Cybernetics}{}{Department of Nonlinear Dynamical Systems and Control Processes}
\cventry{09/2003--06/2008}{MSc in Applied Mathematics and Computer Science with speciality Mathematician, System Programmer}{Lomonosov Moscow State University}{Faculty of Computational Mathematics and Cybernetics}{GPA:~4.47~(out of 5.0)}{Department of Nonlinear Dynamical Systems and Control Processes}
%\cventry{09/1993--06/2003}{Secondary Education: Economic--Mathematical Lyceum \#9}{}{Astana, Kazakhstan; GPA:~4.9~(out of 5.0)}{}{}

\section{Programming skills}
\cvitemwithcomment{}{GNU/Linux, macOS, OpenBSD}{}
\cvitemwithcomment{}{\emph{Programming}}{\emph{TypeScript/JavaScript (ES2021)}, \emph{node.js/Deno}, 
\emph{Go}, \emph{C99}, \emph{C++17}, \emph{Bash}, \emph{Python}, \emph{HTML/CSS/SASS}}
%, \emph{QML}, \emph{Fortran}, \emph{CUDA/OpenCL}, \emph{Simatic~Step~7}, \emph{Assembler}}
\cvitemwithcomment{}{\emph{Frameworks}}{\emph{Qt}, \emph{Vue.js}, \emph{Angular}, \emph{React}, \emph{three.js},  \emph{D3.js}, \emph{express.js}, \emph{Bulma/Bootstrap}, \emph{FastAPI}, \emph{Echo}}
\cvitemwithcomment{}{\emph{DevOps}}{\emph{Docker}, \emph{Kubernetes}, \emph{AWS}, \emph{Azure}, \emph{DigitalOcean}, \emph{GitHub Actions}, \emph{Jenkins}}
%\cvitemwithcomment{}{\emph{IDEs}}{\emph{Visual Studio Code}, \emph{Qt Creator}, \emph{Xcode}, \emph{Visual Studio}, \emph{Simatic Manager}}
\cvitemwithcomment{}{\emph{Bug tracking systems}}{\emph{Jira}, \emph{Redmine}, \emph{GitHub}, \emph{TFS}}
%\cvitemwithcomment{}{\emph{Revision control systems}}{\emph{Git}, \emph{SVN}}
\cvitemwithcomment{}{\emph{Technologies}}{\emph{Redis}, \emph{MongoDB}, \emph{RabbitMQ}, \emph{WebSockets}, \emph{AMQP}, \emph{MQTT}, \emph{SSE}, \emph{OpenAPI/Swagger}, \emph{Traefik/NGINX/HAproxy}, \emph{S3}, \emph{OAuth 2.0}, \emph{Keycloak}, \emph{WebGL}, \emph{JSON-RPC}, \emph{esbuild}, \emph{webpack}, \emph{Vite}, \emph{Poetry}, \emph{pydantic}}
%\cvitemwithcomment{}{\emph{Simulation software}}{\emph{LabVIEW, MATLAB, SimulationX, Vortex, OpenGL, R}}
%\cvitemwithcomment{}{\emph{Industrial protocols and tools}}{\emph{SIMBA}, \emph{Modbus}, \emph{ModSAFE}, \emph{OPC}, \emph{PROFIBUS}, \emph{PROFINET}, \emph{Proficy HMI/SCADA -- CIMPLICITY}}
%\cvitemwithcomment{}{}{}
%\cvitemwithcomment{}{}{}

%\section{PhD Thesis (06/2010--\ldots{})} 
%\cvitem{Thesis}{\emph{Inverse Modelling of Greenhouse Gas Transport over East Asia}} 
%\cvitem{Research supervisors}{Prof.~Joseph H. LaCasce (University of Oslo), Dr.~Andreas Stohl (NILU---Norwegian Institute for Air Research), Prof.~Frode Stordal (University of Oslo)}

%\section{MSc Thesis (09/2003--06/2008)} 
%\cvitem{Thesis}{\emph{Analysis of Scenarios of Transition to Chaos in Dissipative Systems of Nonlinear Ordinary Differential Equations}} 
%\cvitem{Research supervisor}{Prof. N.~A.~Magnitskii (Institute for Systems Analysis of Russian Academy of Sciences)} 

%\section{Research Interests} 
%\cvitem{Subject 1}{Numerical analysis of chaotic systems of differential equations (namely, dissipative systems of nonlinear ODEs, PDEs with running wave solutions, Hamiltonian systems, and conservative systems), and search of scenario of transition to chaos in those systems through cascades of soft bifurcations of stable solutions}
%\cvitem{Subject 2}{Inverse problems in modelling of atmospheric transport} 

\section{Languages}
\cvitemwithcomment{Russian}{Native}{}
\cvitemwithcomment{English}{Advanced}{IELTS 7.5 (L: 7.5; R: 8.5; W: 7.5; S: 7.0) $\sim{}$ C2} 
\cvitemwithcomment{Norwegian}{Intermediate}{Norskprøve 2 (A2)}

%\section{Computer Skills} 
%\cvcomputer{Programming}{GNU/Linux, OS X, Windows}{}{}
%\cvcomputer{}{Programming languages}{}{\emph{C++11}, \emph{Simatic Step 7}, \emph{C99}, \emph{Fortran}, \emph{Bash}, %\emph{CUDA}, \emph{Assembler}}
%\cvcomputer{}{IDEs}{}{\emph{Xcode}, \emph{Microsoft Visual Studio}, \emph{Simatic Manager}, \emph{Borland C++ Builder}, \emph{Borland Delphi}}
%\cvcomputer{}{Simulation software}{}{\emph{LabVIEW, MATLAB, SimulationX, Vortex, OpenGL, R}}
%\cvcomputer{}{Revision control systems}{}{\emph{TFS}, \emph{Git}, \emph{SVN}}
%\cvcomputer{}{Industrial protocols and tools}{}{\emph{SIMBA}, \emph{Modbus}, \emph{OPC}, \emph{PROFIBUS}, \emph{PROFINET}, \emph{Proficy HMI/SCADA -- CIMPLICITY}}
%\cvcomputer{IT}{OS X, Windows, GNU/Linux}{}{\emph{Build PC, reinstall OS, set up network/router/firewall}}
%\cvcomputer{Other}{Typesetting and image editing}{}{\LaTeX, \LyX, \emph{OpenOffice}, \\ \emph{Microsoft Office}, \emph{Adobe Photoshop}}

%\section{Personal}

%\subsection{Personal Qualities} 
%\cvitem{}{\small Disciplined, exact, punctual, responsible, logical, self-restrained, persistent} 

\section{Hobbies}
\cvitem{}{\small IT, science, reading, chess, astronomy, table tennis, squash, volleyball, badminton, cars} 

%\section{References}
%\begin{small}
%\begin{cvcolumns}
	%\cvcolumn{Name}{Prof. Joseph H. LaCasce \\ Kjell Andresen \\ Prof. Nikolai Magnitskii}
	%\cvcolumn[0.32]{Affiliation}{University of Oslo (UiO) \\ University of Oslo (UiO) \\ Institute for Systems Analysis}
 %\cvcolumn[0.41]{Contact details}{\href{mailto:j.h.lacasce@geo.uio.no}{j.h.lacasce@geo.uio.no}; +47\,228\,55955 \\ %\href{mailto:kjell@geo.uio.no}{kjell@geo.uio.no}; +47\,913\,26\,180 \\ \href{mailto:nmag@isa.ru}{nmag@isa.ru}; +7\,499\,1354332}
%\end{cvcolumns}
%\end{small}

%\enlargethispage{15pt}

\section{Select Publications}
%\cvitem{[1]}{\textbf{A. Trebler}, R. L. Thompson, and S. Eckhardt, Estimating Primary and Secondary Sources of Persistent Organic Pollutants Using Inverse Methods, \emph{Atmospheric Environment}, 2013--2015.}
%\cvitem{[2]}{J. Kim, P. Fraser, J. M\"{u}hle, S. Li, A. Manning, \textbf{A. Trebler}, A. Stohl, A. Ganesan, P. Krummel, P. Steele, T. Saito, S. Park, S.-K. Kim, M.-K. Park, T. Arnold, C. Harth, P. Salameh, Y. Yokouchi, R. Weiss, R. Prinn, and K.-R. Kim, Atmospheric verification of reported PFC emissions from the global and East Asian aluminium and semiconductor industries, 2011--2013, in preparation.}
%\cvitem{[2]}{J. Kim, \textbf{A. Trebler}, S. Li, J. M\"{u}hle, S. Park, M. Park, S. Kim, T. Arnold, C. M. Harth, P. Salameh, A. Stohl, R. F. Weiss, and K. Kim, ``Emissions of HFCs in East Asia: Consumption or Production?'', in \emph{AGU Fall Meeting Abstracts}, 2011, \url{http://adsabs.harvard.edu/abs/2011AGUFM.A11E0125K}.}
%\cvitem{[4]}{J. Kim, P. Fraser, J. M\"{u}hle, S. Li, A. Manning, \textbf{A. Trebler}, A. Stohl, A. Ganesan, P. Krummel, P. Steele, T. Saito, S. Park, S. Kim, M. Park, T. Arnold, C. Harth, P. Salameh, Y. Yokouchi, R. Weiss, R. Prinn, and K. Kim, ``Emissions of Tetrafluoromethane and Hexafluoroethane: Balancing Anthropogenic Budgets from Atmospheric Measurements,'' in \emph{Earth System Research Laboratory Global Monitoring Annual Conference}, (Boulder, Colorado), 2012, \url{http://www.esrl.noaa.gov/gmd/annualconference/previous/2012/abs.php?refnum=102-120409-A}.}
\cvitem{[1]}{\textbf{A. Trebler}, A. Stohl, and P. Seibert, ``Identification of Greenhouse Gas Emission Sources Using Analytical Inverse Method,'' in \emph{Algorithmic Analysis of Unstable Problems: Abstracts of the International Conference Dedicated to the Memory of V. K. Ivanov}, Institute of Mathematics and Mechanics, Ural Branch of the Russian Academy of Sciences, Ekaterinburg, Russia, 2011, \url{http://aanz.imm.uran.ru/aanz/AANZ-2011-final.pdf}.}
%\cvitem{[6]}{J. Kim, S. Li, A. Stohl, A. Manning, \textbf{A. Trebler}, S. Park, F. Jin,  S.-K. Kim, M.-K. Park, J. M\"{u}hle, R. Weiss, and K.-R. Kim, ``Measurements of Greenhouse Gases at Gosan (Jeju Island,  Korea) for Regional Analysis of Emissions in East Asia,'' in \emph{AGU Chapman Conference on  Advances in Lagrangian Modeling of the Atmosphere}, Grindelwald, Switzerland, 2011, \url{http://www.empa.ch/plugin/template/empa/*/113661}.}
\cvitem{[2]}{\textbf{Andrey Trebler}, On Cascades of Bifurcations Leading to Chaos in Several Nonlinear Dissipative Systems of ODEs, \emph{Communications in Nonlinear Science and Numerical Simulation}, vol.~15, no.~10, pp.~2974--2986, 2010, \href{http://dx.doi.org/10.1016/j.cnsns.2009.11.019}{doi: 10.1016/j.cnsns.2009.11.019}.}
\cvitem{[3]}{\textbf{Andrey Trebler}, A Transition to Chaos in Rucklidge Model of Double Convection, in \emph{CIMCA '08: Proceedings of the 2008 International Conference on Computational Intelligence for Modelling Control \& Automation}, 2008, IEEE Computer Society, pp.~952--957, \href{http://dx.doi.org/10.1109/CIMCA.2008.46}{doi: 10.1109/CIMCA.2008.46}.}
%\cvitem{[9]}{N. M. Evstigneev, N. A. Magnitskii, and \textbf{A. A. Trebler}, Nonlinear Dynamics of a Laminar--Turbulent Transition in a 3D Problem on the Motion of a Fluid Behind a Ledge, in: \emph{Proceedings of the Conference ``Theory and Practice of the System Analysis''}, Ribinsk (Russia), 2010, in Russian.}
%\cvitem{[10]}{\textbf{Andrey Trebler}, On Classification of Attractors in Nonlinear Dissipative Systems of ODEs, \emph{Dynamics of Heterogeneous Systems, Transitions of the Institute for System Analysis, Russian Academy of Science}, 2009, vol.~44, pp.~123--144, in Russian, \url{http://www.isa.ru/proceedings/images/documents/2009-44/126-146.pdf}.}
%\cvitem{[11]}{\textbf{Andrey Trebler}, On Regular and Singular Attractors in Dissipative Systems of Nonlinear ODEs, in: \emph{Proceedings of the International Multiconference ``Control: Theory and Systems''}, Moscow (Russia), 2009, pp.~32--36, in Russian.}
%\cvitem{[12]}{\textbf{Andrey Trebler}, On Shilnikov Criteria and Modern Theory of Chaos in Dissipative Systems of Nonlinear ODEs, in: \emph{Proceedings of the Second International Conference for Young Mathematicians on Differential Equations and Applications}, Donetsk (Ukraine), 2008, p.~108, in Russian.}
%\enlargethispage{35pt}
%\cvitem{[13]}{\textbf{Andrey Trebler}, Chaos in Some Dissipative Systems of Nonlinear ODEs, in: \emph{Proceedings of the International Conference of Students and Young Scientists ``Lomonosov--2008''}, Astana (Kazakhstan), 2008, pp.~201--203, in Russian.}
\end{document}
